\chapter{Podsumowanie}
\section{Podsumowanie treści}

Rozdział podsumowujący ma na celu zebranie najważniejszych informacji z poprzednich części dokumentu oraz wskazanie kierunków rozwoju. Dzięki rozbudowanej strukturze dokumentu możliwe jest łatwe jego rozszerzanie, modyfikowanie oraz adaptowanie do potrzeb akademickich lub technicznych.

"Granice mojego języka oznaczają granice mojego świata" - Ludwig Wittengenstein \cite{wittgenstein1922}


\subsection{Najważniejsze punkty}
\begin{itemize}
\item Znaczenie przejrzystej struktury oraz logicznego podziału treści.
\item Korzyści płynące z wykorzystania tabel i list w prezentacji informacji.
\item Możliwość wykorzystania dokumentu jako szablonu.
\item Elastyczność tekstu i łatwość rozbudowy.
\end{itemize}

\subsection{Kierunki rozwoju}
\begin{itemize}
\item Dodanie przykładów matematycznych i wzorów LaTeX.
\item Integracja wykresów i diagramów.
\item Stworzenie bibliografii i systemu cytowań.
\item Rozbudowa o rozdziały dotyczące analizy danych.
\item Wprowadzenie treści specjalistycznych dostosowanych do projektu.
\end{itemize}
