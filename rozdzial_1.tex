\chapter{Wprowadzenie}
\section{Wprowadzenie}

Tworzenie rozbudowanego dokumentu tekstowego wymaga starannego przemyślenia zarówno treści, jak i struktury. Współczesne standardy akademickie oraz techniczne kładą duży nacisk na przejrzystość, logiczny układ informacji oraz możliwość łatwej konwersji dokumentu do innych formatów, w tym do LaTeX. System LaTeX, stosowany powszechnie w naukach ścisłych i technicznych, premiuje dokumenty uporządkowane, przygotowane z zastosowaniem hierarchicznego podziału treści oraz elementów takich jak tabele, listy, wzory czy odwołania.

W niniejszym rozdziale omówione zostaną motywacje, jakie towarzyszyły powstaniu tego dokumentu. Przybliżony zostanie również zakres tematyczny, a także znaczenie odpowiednio przygotowanej struktury dla dalszej pracy nad tekstem – niezależnie od tego, czy będzie ona polegała na rozbudowie dokumentu, analizie danych, tworzeniu raportu czy przygotowaniu publikacji naukowej. Rozbudowa treści wprowadzenia pozwala czytelnikowi na zrozumienie kontekstu i celu całego projektu.
"Matematyka jest królową nauk" - Carl Friedrich Gauss \cite{gauss1801}.

\subsection{Cel dokumentu}
Głównym celem dokumentu jest stworzenie wieloelementowego materiału, który może stanowić punkt wyjścia do dalszej pracy w środowisku LaTeX. Dokument został zbudowany w taki sposób, aby zawierał różnorodne komponenty strukturalne: rozdziały i podrozdziały, tabele, listy numerowane i wypunktowane, a także rozbudowane akapity. Dzięki temu może pełnić rolę szablonu, na bazie którego możliwe jest budowanie bardziej zaawansowanych projektów.

\subsection{Zakres tematyczny}
Zakres dokumentu obejmuje przegląd metod organizacji treści zgodnie z najlepszymi praktykami edytorskimi. Omówione zostaną podstawowe i bardziej zaawansowane mechanizmy porządkowania informacji. Dodatkowo poruszona zostanie kwestia pracy z danymi oraz ich prezentacji w sposób czytelny i zgodny z zasadami sztuki.

\begin{figure}

\includegraphics[scale=0.4]{image1.jpg}

\caption{Zrzut ekranu z "The Elder Scrolls V: Skyrim", rok 2011}
\label{figimage}
\end{figure}
